\documentclass{ctexart}%ctexbook,ctexrep
\usepackage{geometry}
\usepackage{fancyhdr}
\usepackage{ragged2e}
\usepackage{fontspec}
\usepackage{xunicode-addon}


% 目录字体设置
\usepackage{subfigure}
\usepackage[subfigure]{tocloft}
\renewcommand{\cftdotsep}{0.5}
\renewcommand{\cftsecleader}{\cftdotfill{\cftdotsep}} % 目录后一行连续的点
\renewcommand{\cftsecfont}{\songti \zihao{-4}}
\renewcommand{\cftsecpagefont}{\songti \zihao{-4}}
% 目录居中
\renewcommand*\contentsname{\hfill \heiti \zihao{-3}  目 \qquad 录 \hfill}

\geometry{left=2.5cm,right=2.5cm,top=3.5cm,bottom=2.5cm}

\setCJKfamilyfont{heitibf}[FakeBold]{SimHei}             %使用STXihei华文细黑加粗字体
\newcommand{\heitibf}{\CJKfamily{heitibf}}
\setCJKfamilyfont{hwxh}{STXihei}             %使用STXihei华文细黑字体
\newcommand{\huawenxihei}{\CJKfamily{hwxh}}
\setCJKfamilyfont{hwxhbf}[FakeBold]{STXihei}             %使用STXihei华文细黑加粗字体
\newcommand{\huawenxiheibf}{\CJKfamily{hwxhbf}}
\setCJKfamilyfont{hwxk}{STXingkai}             %使用STXingkai华文行楷字体
\newcommand{\huawenxingkai}{\CJKfamily{hwxk}}

\newenvironment{abstract1}[0]{}{}%可选参数,默认值为摘要
{}

%=================设置章节标题格式==================
\ctexset{
	section={
		%format用于设置章节标题全局格式,作用域为标题和编号
		%字号为小三,字体为黑体,左对齐
		%+号表示在原有格式下附加格式命令
		format+ = \zihao{-3} \heiti \justifying \linespread{1.5},
		%name用于设置章节编号前后的词语
		%前、后词语用英文状态下,分开
		%如果没有前或后词语可以不填
		name = {,\quad },
		%number用于设置章节编号数字输出格式
		%输出section编号为中文
		number = \arabic{section},
		%beforeskip用于设置章节标题前的垂直间距
		%ex为当前字号下字母x的高度
		%基础高度为1.0ex,可以伸展到1.2ex,也可以收缩到0.8ex
		beforeskip = 1.0ex plus 0.2ex minus .2ex,
		%afterskip用于设置章节标题后的垂直间距
		afterskip = 11pt,
		%aftername用于控制编号和标题之间的格式
		%\hspace用于增加水平间距
		aftername = \hspace{0pt}
	},
	subsection={
		format+ = \zihao{4} \heiti \justifying \linespread{1.5},
		%仅输出subsection编号且为中文
		number = {\arabic{section}.\arabic{subsection}},
		name = {,\quad },
		beforeskip = 7pt,
		afterskip = 1.0ex plus 0.2ex minus .2ex,
		aftername = \hspace{0pt}
	},
	subsubsection={
		%设置对齐方式为居中对齐
		format+ = \zihao{-4} \heiti \justifying \linespread{1.5},
		%仅输出subsubsection编号,格式为阿拉伯数字,打字机字体
		number = \arabic{section}.\arabic{subsection}.\arabic{subsubsection},
		name = {,\quad },
		beforeskip = 7pt,
		afterskip = 1.0ex plus 0.2ex minus .2ex,
		aftername = \hspace{0pt}
	},
	paragraph={
		format = \songti \zihao{-4} \justifying \linespread{1.25},
		indent={30pt}
	}
}

\newenvironment{paar}[0]
	{\par \songti \zihao{-4}  \justifying \linespread{1.25} \setlength{\parindent}{24pt} }{\par}
\newenvironment{seq1}[1]
	{\par \songti \zihao{-4}  \justifying \linespread{1.25} \setlength{\parindent}{24pt} ({#1})  }{\par}
\newenvironment{seq2}[1]
	{\par \songti \zihao{-4}  \justifying \linespread{1.25} \setlength{\parindent}{24pt} \textcircled{#1} }{\par}
\newenvironment{jielun}[0]
	{\par \heiti \zihao{-3} \center \linespread{1.5} 结\quad 论(设计类为设计总结) \vspace{11pt}\par
	\songti \zihao{-4} \justifying \linespread{1.25} 
	}{ \par}
\newenvironment{wenxian}[0]
	{\par \heiti \zihao{-3} \center \linespread{1.5} 参\quad 考\quad 文\quad 献 \vspace{11pt}\par
	\songti \zihao{5} \justifying \linespread{1.25} 
	}{\par}
\newenvironment{fulu}[1]
	{\par \heiti \zihao{-3} \center \linespread{1.5} 附\quad 录\quad #1 \vspace{11pt}\par
	\par \songti \zihao{-4}  \justifying \linespread{1.25} \setlength{\parindent}{24pt} 
	}{\par}
\newenvironment{jilu}[0]
	{\par \heiti \zihao{-3} \center \linespread{1.5} 修\quad 改\quad 记\quad 录 \vspace{11pt}\par
	\par \songti \zihao{-4}  \justifying \linespread{1.25} \setlength{\parindent}{24pt} 
	}{\flushright 记录人(签字):\qquad \qquad \qquad \\ 指导教师(签字):\qquad \qquad \qquad \par}	
\newenvironment{zhixie}[0]
	{\par \heiti \zihao{-3} \center \linespread{1.5} 致\quad 谢 \vspace{11pt}\par
	\par \songti \zihao{-4}  \justifying \linespread{1.25} \setlength{\parindent}{24pt} 
	}{\par}
 
\begin{document}
	\songti \zihao{-4}
	~\\
	\songti \center \zihao{-1}  大连理工大学本科毕业论文(设计)\\
	\songti \zihao{-4}
	~\\
	~\\
	\huawenxiheibf \zihao{2} \linespread{1.25}  基于相机相应不一致性的来源鉴别研究 \\ 
	\rmfamily \zihao{3} Source Identification Based on Camera Corresponding Inconsistencies \\
	\songti \zihao{-4}
	~\\
	~\\
	~\\
	~\\
	~\\
	~\\
	~\\
	~\\
	~\\	
	\songti \zihao{-3}
	学院(系):\underline{\quad \ \, 信通学院\quad \ \,  }  \\
	专业:\underline{\quad \quad 电子信息工程\quad \quad }    \\
	学生姓名:\underline{\quad \quad \quad 游轩\quad \quad \quad }      \\
	学号: \underline{\quad \quad \quad 201981261\quad \quad \quad }    \\
	指导教师:\underline{\quad \quad \quad 王波\quad \quad \quad }  \\
	评阅教师:\underline{\quad \quad (待定)\quad \quad }  \\
	完成日期:\underline{\number\year 年 \number\month 月 \number\day 日}    \\
	\huawenxingkai \zihao{-2}
	~\\
	~\\
	~\\
	大连理工大学\\
	\rmfamily \zihao{-4} Dalian University of Technology
	\clearpage

	\pagestyle{fancy}
	\fancyhf{}
	\fancyhead[CO, CE]{\center \songti \zihao{5} 基于相机相应不一致性的来源鉴别研究}
	%页眉线宽,设为0可以去页眉线
	\renewcommand{\headrulewidth}{0.1mm} 
	\huawenxiheibf \zihao{2} \center 原创性声明 \\
	\songti \zihao{-3} \justifying \linespread{1.25} 本人郑重声明:本人所呈交的毕业论文(设计),是在指导老师的指导下独立进行研究所取得的成果。毕业论文(设计)中凡引用他人已经发表或未发表的成果、数据、观点等,均已明确注明出处。除文中已经注明引用的内容外,不包含任何其他个人或集体已经发表或撰写过的科研成果。对本文的研究成果做出重要贡献的个人和集体,均已在文中以明确方式标明。\\
	~\\
	本声明的法律责任由本人承担。\\
	~\\
	~\\
	作者签名:\quad \quad \quad \quad \quad \quad \quad \quad \quad \quad  日  期:\today
	\clearpage
	
	\huawenxiheibf \zihao{2} \center \linespread{1.5} 关于使用授权的声明 \\ 
	\vspace{11pt}
	\songti \zihao{-3} \justifying  \linespread{1.25} 本人在指导老师指导下所完成的毕业论文(设计)及相关的资料(包括图纸、试验记录、原始数据、实物照片、图片、录音带、设计手稿等),知识产权归属大连理工大学。本人完全了解大连理工大学有关保存、使用毕业论文(设计)的规定,本人授权大连理工大学可以将本毕业论文(设计)的全部或部分内容编入有关数据库进行检索,可以采用任何复制手段保存和汇编本毕业论文(设计)。如果发表相关成果,一定征得指导教师同意,且第一署名单位为大连理工大学。本人离校后使用毕业毕业论文(设计)或与该论文直接相关的学术论文或成果时,第一署名单位仍然为大连理工大学。\\
	~\\
	~\\
	论文作者签名:\qquad \qquad \qquad  \qquad \qquad   日  期:\today    \\
	指导老师签名:\qquad \qquad \qquad  \qquad \qquad   日  期:\today    \\
	\clearpage
	
	\pagenumbering{Roman}
	\fancyfoot[CO, CE]{\songti \zihao{-5} -\thepage -}
	\begin{abstract1}
	\heiti \zihao{-3} \linespread{1.5} \center 摘\qquad 要\\ 	\vspace{11pt}
	\setlength{\parindent}{24pt}
	\songti \zihao{-4} \justifying \linespread{1.25} “摘要”是摘要部分的标题,不可省略。\\
	标题“摘要”选用模板中的样式所定义的“标题1”,再居中;或者手动设置成字体:黑体,居中,字号:小三,1.5倍行距,段后11磅,段前为0。\par 
	摘要是毕业论文(设计)的缩影,文字要简练、明确。内容要包括目的、方法、结果和结论。单位采用国际标准计量单位制,除特别情况外,数字一律用阿拉伯数码。文中不允许出现插图。重要的表格可以写入。\par 
	摘要正文选用模板中的样式所定义的“正文”,每段落首行缩进2个汉字;或者手动设置成每段落首行缩进2个汉字,字体:宋体,字号:小四,行距:多倍行距 1.25,间距:段前、段后均为0行,取消网格对齐选项。\par 
	摘要篇幅以一页为限,字数限500字以内。\par 
	摘要正文后,列出3-5个关键词。“关键词:”是关键词部分的引导,不可省略。关键词请尽量用《汉语主题词表》等词表提供的规范词。\par 
	关键词与摘要之间空一行。关键词词间用分号间隔,末尾不加标点,3-5个;黑体,小四,加粗。关键词整体字数限制在一行。\par
	~\\
	\heitibf \zihao{-4} \justifying 关键词:写作规范;排版格式;毕业论文(设计)
	\end{abstract1}
	\addcontentsline{toc}{section}{摘\qquad 要}\tolerance=500 %将摘要放进目录
	\clearpage
	
	\begin{abstract1}
	\setmainfont{Times New Roman} \zihao{-3} \bfseries \center \linespread{1.25} Source Identification Based on Camera Corresponding Inconsistencies\\
	\heiti \zihao{-3} ~\\
	\setmainfont{Times New Roman} \zihao{-3} \center \linespread{1.5} Abstract\\
	\vspace{11pt}
	\setlength{\parindent}{24pt}
	\setmainfont{Times New Roman} \zihao{-4} \justifying \linespread{1.25} Times New Roman\par
	Times New RomanTimes New RomanTimes New RomanTimes New RomanTimes New RomanTimes New RomanTimes New RomanTimes New RomanTimes New RomanTimes New RomanTimes New Roman\par 
	Times New RomanTimes New RomanTimes New RomanTimes New RomanTimes New RomanTimes New RomanTimes New RomanTimes New RomanTimes New Roman\par 
	~\\
	\setmainfont{Times New Roman} \zihao{-4} \bfseries \justifying  Key Words:Write Criterion;Typeset Format;Graduation Project (Thesis)
	\end{abstract1}
	\addcontentsline{toc}{section}{Abstract}\tolerance=500 %将Abstract放进目录	
	\clearpage	

	\tableofcontents
	\clearpage

	\pagenumbering{arabic}
	\setcounter{page}{1}	
	\fancyfoot[CO,CE]{\songti \zihao{-5}-\thepage -}
	\begin{abstract1}
	\heiti \zihao{-3} \center \linespread{1.5} 引 \qquad 言\\
	\vspace{1ex}
	\songti \zihao{-4} \justifying \linespread{1.25}
	理工文科所有专业本科生的毕业论文(设计)都应有“引言”的内容。如果引言部分省略,该部分内容在正文中单独成章,标题改为文献综述,用足够的文字叙述。从引言开始,是正文的起始页,页码从1开始顺序编排。\par
	针对做毕业设计:说明毕业设计的方案理解,阐述设计方法和设计依据,讨论对设计重点的理解和解决思路。\par
	针对做毕业论文:说明论文的主题和选题的范围;对本论文研究主要范围内已有文献的评述;说明本论文所要解决的问题。建议与相关历史回顾、前人工作的文献评论、理论分析等相结合。
注意:是否如实引用前人结果反映的是学术道德问题,应明确写出同行相近的和已取得的成果,避免抄袭之嫌。注意不要与摘要内容雷同。\par
	书写格式说明:\par
	标题“引言”选用模板中的样式所定义的“引言”;或者手动设置成字体:黑体,居中,字号:小三,1.5倍行距,段后1行,段前为0行。\par
	引言的字数在3000字左右(毕业设计类引言可适当调整为800字左右)。引言正文选用模板中的样式所定义的“正文”,每段落首行缩进2字;或者手动设置成每段落首行缩进2字,宋体,小四,多倍行距 1.25,段前、段后均为0行,取消网格对齐选项。\par
	\end{abstract1}
	\addcontentsline{toc}{section}{引 \qquad 言}\tolerance=500 %将引 \qquad 言放进目录
	\clearpage
	
	\section{正文格式说明}
	正文是毕业论文(设计)的主体,是毕业论文或工程设计说明书的核心部分。要求学生运用所学的数学、自然科学、工程基础和专业知识解决复杂问题的能力,能够针对问题设计解决方案,在设计环节中体现创新意识,并考虑社会、健康、安全、法律、文化、环境以及社会可持续发展等因素;要着重反映毕业设计或论文的工作,要突出毕业设计的设计过程、设计依据及解决问题的方法;毕业论文重点要突出研究的新见解,例如新思想、新观点、新规律、新研究方法以及新结果等。\par
	\subsection{hello你好啊world}
	\subsubsection{h你好啊i}
	%\paragraph{nihao\\}

	\begin{paar}
		新时代青年\\dfsdsfsdfdswe的第三方对手的防守的发射点上的\\
	\end{paar}
	\begin{seq2}{1}
	新时代新青年
	\end{seq2}
	\begin{paar}
		新时代青年\\dfsdsfsdfdswe的第三方对手的防守的发射点上的\\
	\end{paar}	
	\begin{seq1}{2}
	新时代新青年
	\end{seq1}
	\begin{seq2}{3}
	新时代新青年
	\end{seq2}
	\begin{seq2}{4}
	新时代新青年
	\end{seq2}
	\section{sophi你好啊e}
	\subsection{benea你好啊th}
	\subsubsection{grou你好啊nd}
	
	\clearpage
	
	\begin{jielun}
	
	\end{jielun}
	\addcontentsline{toc}{section}{结 \qquad 论}\tolerance=500 %将结 \qquad 论放进目录
	\clearpage
	
	\begin{wenxian}
	
	\end{wenxian}
	\addcontentsline{toc}{section}{参\quad 考\quad 文\quad 献}\tolerance=500 %将参\quad 考\quad 文\quad 献放进目录
	\clearpage
	
	\begin{fulu}{A}
	
	\end{fulu}
	\addcontentsline{toc}{section}{附录A\quad 附录内容名称 }\tolerance=500 %将附\qquad 录\qquad A放进目录
	\clearpage
	
	\begin{jilu}
	
	\end{jilu}
	\addcontentsline{toc}{section}{修改记录}\tolerance=500 %将修\qquad 改\qquad 记\qquad 录放进目录
	\clearpage
	\begin{zhixie}
	
	\end{zhixie}
	\addcontentsline{toc}{section}{致\qquad 谢}\tolerance=500 %将致\qquad 谢放进目录
\end{document}